%!TEX root = ../dissertation.tex
\chapter{Conclusion}
\label{conclusion}

established a framework for the fast and efficient computation of complex smRNA:TF:gene interactions, tissue specificity

(re-)established an \emph{in vitro} human cellular model for the study of cholinergic neurons with regard to sex specific phenomena, and, particularly, the effect of neurokine differentiation on cholinergic processes

in this context, elucidated the potential impact of small RNA dynamics in psychiatric diseases

identified pertinent mechanisms in the response to stroke of blood-borne human cells, and interactions between small and large transcripts in these cells

in multiple instances identified a neuro-immune axis connecting neurokine mechanisms and cholinergic signalling

explored the feasibility and practicality of feedforward loop analyses, in general and on the example of CD14$^+$ monocytes in post-stroke blood samples

took a first step in utilising advanced network approaches for data science in the life sciences, for instance by utilising »smart« dimensionality reduction
