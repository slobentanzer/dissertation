%!TEX root = ../dissertation.tex
\chapter{Conclusion}
\label{conclusion}
The study of small RNA dynamics is greatly facilitated by modern bioinformatic methods that enable an understanding of complex transcriptional events in unprecedented detail. While the application and integration of the diverse sources characterising interactions and the participating molecules still are prone to error and misinterpretation, the field advances in giant steps. In summary, the dissertation here presented contributes to the field the following findings:

\begin{itemize}[noitemsep, leftmargin=.5cm, label={\tiny\raisebox{.5ex}{\textbullet}}]
\item Establishment of a framework for the fast and efficient computation of complex interactions between smRNAs, transcription factors, and target genes, including high-resolution tissue specific effects, and on the scale of the whole genome and all miRNAs and tRFs simultaneously;

\item (Re-)establishment of an \emph{in vitro} human cellular model for the study of cholinergic neurons with regard to sex specific phenomena, and, particularly, the effect of neurokine differentiation on cholinergic processes;

\item In this context, elucidation of the potential impact of small RNA dynamics in psychiatric diseases;

\item Identification of pertinent mechanisms in the response to stroke of blood-borne human cells, and interactions between small and large transcripts in these cells;

\item Identification of a neuro-immune axis connecting neurokine mechanisms and cholinergic signalling in multiple instances;

\item Exploration of the feasibility and practicality of feedforward loop analyses, in general and in an example of CD14$^+$ monocytes in post-stroke blood samples;

\item Description of a first step in utilising advanced network approaches for data science in the life sciences, for instance by implementing a »smart« dimensionality reduction.
\end{itemize}
