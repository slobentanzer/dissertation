%!TEX root = ../dissertation.tex
\chapter{Required Documents} 
\label{appendix:req-doc}

\section{Declarations}
Except where stated otherwise by reference or acknowledgment, the work presented was generated by myself under the supervision of my advisors during my doctoral studies. All contributions from colleagues are explicitly referenced in the thesis. The material listed below was obtained in the context of collaborative research:

Figure \ref{fig:}: title, Geula Hanin (formerly Hebrew University of Jerusalem, now University of Oxford), her contribution: cell culture experiments generating data for panels B-D, my own contribution: 

Chapter Four, data generation by sequencing of patient blood samples, sample generation (RNA isolation): Bettina Nadorp, formerly Hebrew University of Jerusalem, now University of ?

All RNA sequencing was carried out in the National Center for Genomic Technologies, a Core Facility in the Alexander Silberman Institute of Life Sciences of the Hebrew University, Jerusalem, and with the help of Dr. Estelle Bennett (Soreq lab).

other contributions e.g. text or equations

Whenever a figure, table or text is identical to a previous publication, it is stated explicitly in the thesis that copyright permission and/or co-author agreement has been obtained.

The following parts of the thesis have been previously published:
- Chapter “XYZ”
- Figure(s) “XYZ”
- Table(s) “XYZ”

\newpage

\section{Summary in German Language}
Die vorliegende Dissertation beschäftigt sich mit der Charakterisierung und Analyse von kleinen nicht-codierenden RNA-Molekülen im Kontext der Regulation cholinerger Prozesse in neuronalen und nicht-neuronalen Geweben. Durch die geometrische Komplexität der Interaktion von kleinen RNAs (smRNAs) mit codierenden Transkripten in der Zelle, die in der Art eines »viele-zu-viele« Netzwerkes organisiert ist, kann man eine umfassende Analyse der Dynamiken der smRNA-Reg"-u"-la"-tion nur mit bioinformatischer Hilfe unternehmen. Diese Dissertation etabliert eine bio"-in"-for"-ma"-ti"-sche Infrastruktur für solche Analysen (unter dem Namen \emph{miRNeo}), und diskutiert dann deren Anwendung an zwei klinischen Beispielen: 1) die neuronale Rolle von cholinerger smRNA-Reg"-u"-la"-tion in den psychiatrischen Erkrankungen Schizophrenie und Bipolare Störung,\cite{Lobentanzer2019a} und 2) die Relevanz cholinerg-assoziierter smRNAs in immunologischen Reflexen im Blut von Schlaganfall-Patienten.\cite{Winek2020} Im Hinblick auf die steigende Relevanz von therapeutischen Mechanismen im smRNA-Bereich gibt die Dissertation einen Ausblick auf die bioinformatisch gestützte Arzneistoff-Entwicklung, besonders in Beachtung der smRNA-typischen Funktionsweise und multiplen Interaktionen mit codierenden Genen und Transkriptionsfaktoren (TFs).

\subsection{Einführung}
Cholinerge Systeme sind unerlässlich für die physiologische Funktion des Säugetier-Organismus (Kapitel \ref{sec:intro:cholinergic}). Sie zeichnen sich aus durch die Verwendung des Signalmoleküls Acetylcholin (ACh), der erste Neurotransmitter, für den eine derartige Funktion nachgewiesen wurde.\cite{Dale1914,Loewi1921,Dale1929} Cholinerge Systeme sind im zentralen Nervensystem hochrelevant, und hier vor allem in einer modulatorischen Rolle. Cholinerge Projektionen aus den acht primären cholinergen Kernen Ch1-Ch8 üben Kontrolle über weite Bereiche des Mittelhirns und Cortex aus.\cite{Mesulam1984} Zusätzlich existieren cholinerge Interneurone mit kurzen Projektionen in mehreren Hirngebieten, zuvorderst im Striatum, und, erst seit kurzem bekannt, im Cortex.

Cholinerge Systeme sind, parallel zu ihrer wichtigen Rolle in physiologischen Prozessen, auch beteiligt an verschiedenen Krankheitsprozessen (Kapitel \ref{sec:intro:diseases}). Ätiologisch ist eine kausale Rolle von ACh in mehreren zentralnervösen und peripheren Erkrankungen nachgewiesen, beispielsweise in der Alzheimer-Demenz, Schizophrenien und Bipolaren Störungen, sowie in peripheren Immunreflexen, auch im Zusammenhang mit Schlaganfall. In der Alzheimer-Erkrankung stellt die Hemmung des ACh-abbauenden Enzyms Acetylcholinesterase (AChE) das dominierende Therapieprin"-zip dar. Allerdings ist die aktuelle Therapie begrenzt auf Symptomatik und eventuelle Ver"-lang"-sa"-mung des geistigen Verfalls. In psychiatrischen Erkrankungen, die oft multigenetische Ätiologien aufweisen, ist die Beteiligung von cholinergen Systemen komplex und Gegenstand der aktiven Forschung. Gleichermaßen ist der Einfluss cholinerger Signale im Bereich der Immunität noch weitgehend ungeklärt. Es gibt deutliche Hinweise auf die Beteiligung von ACh an der Regulation von T-Zellen, Monozyten/Makrophagen, und an deren Mobilisierung aus der Milz. Die genauen Mechanismen sind allerdings noch kontrovers diskutiert, beispielsweise verfügt die Milz nicht über direkte vagale Innervation, was zu der Hypothese geführt hat, dass die Ausschüttung von ACh in diesem Fall von lokalen Immunzellen unter Stimulation durch den Sympathikus stattfindet (der die Milz umfassend innerviert).

Cholinerge Systeme, genauer, die Zellen der cholinergen Systeme, sind charakterisiert durch Ihre Fähigkeit zur Synthese und Freisetzung von ACh, und damit durch ihren transkriptionellen Phenotyp. Es gibt auch deutliche Hinweise, dass das Vorhandensein des spezifischen Cholintransporters (High Affinity Choline Uptake, HACU, Gensymbol SLC5A7) mit dem cholinergen Phenotyp der betrachteten Zelle korreliert. Kleine RNAs (smRNAs), deren bekannteste Vertreter momentan microRNAs (miRNAs) und tRNA-Fragmente (tRFs) sind, üben auf die zelluläre Ebene des Phenotyps, bestimmt durch mRNA, kontrollierende Funktion aus (Kapitel \ref{sec:intro:mirna} \& \ref{sec:intro:trfs}). miRNAs sind in ihrem reifen Stadium kurze (etwa 18-22 Basen), einzelsträngige RNA-Moleküle, die einen Translationsstopp oder den Abbau von einzelnen mRNAs veranlassen können. Sie nutzen dafür einen Proteinkomplex, der unter dem Namen RISC (RNA-induced silencing complex) bekannt ist, und dienen als »Zielmolekül«: durch Komplementarität mit einer kurzen Sequenz der Ziel-mRNA leiten sie den RISC zur mRNA. tRFs wurden erst vor kurzem als Kontrollmoleküle identifiziert, obwohl ihre Existenz schon länger bekannt ist. Ihre Rolle ist bisher weniger klar als die der miRNAs. Teils wurde für einige tRFs ein miRNA-Mechanismus nachgewiesen (RISC-vermittelter Translationsstopp), aber in anderen Fällen interagierten die tRFs mit anderen funktionalen Bestandteilen von Zellen, beispielsweise RNA-bindenden Proteinen.

Durch eine komplexe »viele-zu-viele«-Organisation ist das Netzwerk der smRNA-Kon"-trol"-le chol"-in"-er"-ger Prozesse eine Struktur, die ohne die Hilfe von informatisch gestützten Systemen vom Menschen nicht erfasst werden kann. Im Folgenden soll ein Ansatz beschrieben werden, die Struktur und Dynamik von smRNAs in cholinergen Systemen sichtbar zu machen; dieser Ansatz soll jedoch nicht auf cholinerge Systeme begrenzt bleiben, sondern ist mit dem Ziel entwickelt, jedes transkriptionelle System in jeder Zelle untersuchen zu können.

\subsection{miRNeo - eine Infrastruktur für smRNA-Dynamik}
Die Entwicklung und Pflege der interaktiven Datenbank \emph{miRNeo} macht einen großen Teil der wissenschaftlichen Arbeit an dieser Dissertation aus, und zieht sich über die gesamte Dauer der Promotion (Kapitel \ref{sec:database}). Der Grundgedanke hinter \emph{miRNeo} ist die Darstellung eines komplexen biologischen Regulationsnetzwerkes in einer digitalen Form die der biologischen Organisation entspricht. Digitale Datensammlungen, insbesondere Datenbanken, sind geprägt durch tabellare Organisation; die Daten werden in Zeilen und Spalten gesammelt, geordnet, und abgerufen. In vielen Fällen ist dieses Vorgehen kein Grund zur Kritik, denn entweder sind die Ressourcen mehr als ausreichend um eine effektive Nutzung zu ermöglichen, oder die natürliche Struktur der Daten ist ohnehin tabellar, beispielsweise ein Telefonbuch. Die Interaktion von RNA-Molekülen im epitranskriptionellen Kontext jedoch ist geprägt von einer hohen Vereinzelung (die meisten potentiellen Interaktionen existieren nicht) und einer Organisationsstruktur die sich durch »viele-zu-viele« beschreiben lässt: eine miRNA reguliert mehrere (teils hunderte) mRNA-Transkripte, und ein mRNA-Transkript wird reguliert von mehreren miRNAs.

Diese strukturelle Organisation birgt eine Problematik für althergebrachte, tabellare digitale Infrastruktur: das Abfragen eines komplexen Zusammenhangs (beispielsweise die Zusammenarbeit von TFs, smRNAs, und mRNAs im cholinergen System) kann mehrere hundert oder tausend einzelne Schritte enthalten und benötigt viel Zeit, Platz, und Rechenleistung. \emph{miRNeo} dagegen basiert auf einem Graph-basierten Datenbankschema: die Einträge werden repräsentiert von Knoten (nodes) und Verbindungen zwischen diesen Knoten (edges). Diese Organisation entspricht der biologischen Organisation genauer als ein tabellares Format es könnte. Durch die technische Implementation, im Rahmen der öffentlichen Datenbanksoftware »Neo4j«, können so Rechenleistung, Platz, und Zeit gespart werden.

Die \emph{miRNeo}-Datenbank enthält in ihrer jetzigen Form Informationen über miRNA-Bindeverhalten auf der Basis sowohl von experimentell validierten als auch von bioinformatisch vorhergesagten miRNA-mRNA-Interaktionen. Durch die Integration verschiedener Quellen verfügt jede miRNA-mRNA-Verbindung über eine Wertung, die sich im Zahlenraum zwischen 3 und 20 bewegt. Durch die Verwendung einer Mindest-Wertung (in den durchgeführten Analysen meist 6 oder 7) kann das untersuchte Netzwerk nach Qualität der Information gefiltert werden. \emph{miRNeo} enthält außerdem eine Sammlung von TF-Regulationskreisläufen (»regulatory circuits«), die für den menschlichen Organismus gewebespezifisch die Ermittlung der Aktivität von TFs gegenüber einzelnen Genen erlauben.\cite{Marbach2016} Außerdem enthält die Datenbank eigene Vorhersagen über die miRNA-ähnliche Interaktion von tRFs (mittels TargetScan), und verbindet mehrere Gen-Nomenklatursysteme um die Integration und Translation zwischen Datensätzen zu ermöglichen (und deren Fehleranfälligkeit zu reduzieren).

\emph{miRNeo} ist aber nicht nur ein Datenspeicher; Neo4j bietet mit der Sprache »CYPHER« die Möglichkeit eines effizienten Datenabrufs von komplexen Zusammenhängen innerhalb des Datenbank-Netzwerkes. Beispielsweise können in einem Arbeitsschritt sämtliche Interaktionen von smRNAs, die das cholinerge System betreffen, abgerufen werden, und verbunden werden mit den TFs, die in einem bestimmten Gewebe (oder mehreren bestimmten Geweben) den cholinergen Phänotyp steuern. Dieser Vorgang, der in einem traditionellen tabellaren System eine Kombination mehrerer Abrufe und nachfolgende Verarbeitung der großen Datenmengen im Arbeitsspeicher erfordern würde, kann durch die hohe Ähnlichkeit der biologischen Daten und der Organisation dieser Daten in \emph{miRNeo} deutlich effizienter, und damit zeitsparender gestaltet werden. Dies is vor Allem von Vorteil, wenn die durchgeführten Analysen ein hohes Maß an Komplexität zeigen, beispielsweise die Ergründung von smRNA:TF:Gen »feedforward«-Schleifen (Kapitel \ref{sec:stroke:ffl}).

\subsection{Cholinerge smRNA-Dynamik in Neuronen}
Cholinerge Neurone sind definiert durch ihre Expression von cholinergen Marker-Genen. Dies sind im engeren Sinne die Cholin-Acetyltransferase (ChAT), die zur Synthese von ACh benötigt wird, und der vesikuläre ACh-Transporter (vAChT, Gensymbol SLC18A3), der die neuronale ACh-Frei"-setz"-ung ermöglicht. Außerdem ist der hochaffine Cholintransporter (CHT-1, Gensymbol SLC5A7) fast ausschließlich in cholinergen Nervenzellen zu finden. Neue Entwicklungen auf dem Gebiet des Einzelzell-Sequencing ermöglichen eine hochauflösende transkriptionelle Darstellung des zentralen Nervensystems. In Anwendung öffentlich zugänglicher Daten gewonnen aus murinem und menschlichem Gewebe fand eine Charakterisierung der Expressionslandschaft zentraler cholinerger Neurone statt (Kapitel \ref{sec:cellculture:singlecell}). Auf Basis dieser Daten, genauer, der Co-Expression von cholinergen Genen und Rezeptoren für Neurokin-Signale (siehe unten), fand eine Bewertung der Eignung der Zellkultur-Modelle LA-N-2 und LA-N-5 statt. Diese menschlichen, neuronalen Zellen differenzieren nämlich unter dem Einfluss von Neurokinen zu Neuronen cholinergen Typs, erkennbar an der ansteigenden Expression von ChAT und vAChT (Kapitel \ref{sec:cellculture:model}). Neurokine sind eine Familie von Zytokin-Typ-Signalmolekülen, die ihre Funktion an Rezeptoren der Familie vom gp130-Typ (Gensymbol IL6ST) ausüben. Die bekanntesten Neurokine sind Interleukin (IL)-6, CNTF (ciliary neurotrophic factor), und LIF (leukaemia inhibiting factor).

Zur Untersuchung der smRNA-Dynamik in den Prozessen die zur cholinergen Entwicklung dieser Zellen unter Neurokin-Einwirkung führen, wurden die Zellen unter der Behandlung mit CNTF zu mehreren Zeitpunkten im Differenzierungsprozess einer RNA-Analyse durch Sequenzierung aller kleinen RNAs unterzogen. RNA wurde isoliert, aus LA-N-2 und LA-N-5 jeweils zu den Zeitpunkten 30 Minuten, 60 Minuten, 2 Tage, und 4 Tage, und gegen unbehandelte Kontrollen verglichen (Kapitel \ref{sec:cellculture:de}). Im Resultat fanden sich 490 teils drastisch veränderte miRNAs zu verschiedenen Zeitpunkten und in einer oder beiden der Zelllinien. 

\subsection{Cholinerge smRNA-Dynamik in Immunzellen}


\newpage

\section{Publications}

\subsection{Peer-reviewed Manuscripts}
\noindent Lobentanzer Cell Reports (2019)

\noindent Stein

\subsection{Book Chapters}
\noindent Lobentanzer \& Klein (2019)

\subsection{Manuscripts with Preprint Status}
\noindent Covid

\noindent Preprints

\subsection{Manuscripts in Submission}
\noindent Kasia

\noindent Tri

\subsection{Invited Manuscripts}
\noindent STAR Protocols

\subsection{Presentations}
\noindent Hebrew University, Silberman Institute Invited (Jerusalem)

\noindent ELSC Invited (Ein Gedi)

\noindent ISCM Invited (Rehovot)

\newpage

\section{Curriculum Vitae (Academic, Including Teachers)}