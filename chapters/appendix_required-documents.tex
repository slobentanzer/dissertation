%!TEX root = ../dissertation.tex
\chapter{Required Documents} 
\label{appendix:req-doc}

\section{Declarations}
\noindent \hangindent=.6cm Except where stated otherwise by reference or acknowledgment, the work presented was generated by myself under the supervision of my advisors during my doctoral studies. The material listed below was obtained in the context of collaborative research:

\noindent \hangindent=.6cm Figure \ref{fig:smallnet}: »The Cholinergic/Neurokine Interface«, Geula Hanin (formerly Hebrew University of Jerusalem, now University of Cambridge), her contribution: cell culture experiments generating data for panels B-D (Section \ref{sec:cellculture:mir125}), my own contribution: data generation and analysis for panel A, visual representation of the entire figure.

\noindent \hangindent=.6cm Chapter 4: sequencing of patient blood samples, sample generation (RNA isolation from patient blood): Bettina Nadorp, formerly Hebrew University of Jerusalem, now NYU Langone Health. All RNA sequencing described in this dissertation was carried out in the National Center for Genomic Technologies, a Core Facility in the Alexander Silberman Institute of Life Sciences of the Hebrew University, Jerusalem, and with the help of Dr. Estelle Bennett (Soreq lab). The work presented in chapter four was done in close collaboration with Kasia Winek and Hermona Soreq of the Soreq lab at the Silberman Institute of Life Sciences, Hebrew University, Jerusalem. Starting with alignment of sequenced reads (Section \ref{sec:stroke:alignment}), all analyses were performed by myself.

\noindent The following parts of the thesis have been previously published:
\begin{itemize}[noitemsep, leftmargin=.5cm, label={\tiny\raisebox{.5ex}{\textbullet}}, topsep = 0pt]
\item Chapter three has in large parts been described in Lobentanzer \emph{et al.}\cite{Lobentanzer2019a} This includes parts of Figures \ref{fig:singlecell}, \ref{fig:time-dose-timeline}, \ref{fig:cc-cor-de-perm}, \ref{fig:mir-de-fam-go}, \ref{fig:bignet}, \ref{fig:scz-bd-go}, and \ref{fig:smallnet}.
\item Figure \ref{fig:bbb} has been published in Lobentanzer \& Klein.\cite{Lobentanzer2019b}
\item Sections \ref{sec:stroke:descriptive-methods}, \ref{sec:stroke:descriptive}, \ref{sec:stroke:celltypes}, and \ref{sec:stroke:circuits} of Chapter four have been partly described in Winek \emph{et al.}\cite{Winek2020} This includes parts of Figures \ref{fig:cholino-ecdfs}, \ref{fig:stroke-de-tsne}, \ref{fig:celltype-presence}, \ref{fig:heatmaps-small}, \ref{fig:tsne-large}, \ref{fig:tsne-small}, and \ref{fig:smrna-tf-network-fractions}.
\end{itemize}

\newpage

\section{Summary in German Language}

\selectlanguage{ngerman}

Die vorliegende Dissertation beschäftigt sich mit der Charakterisierung und Analyse von kleinen nicht-codierenden RNA-Molekülen im Kontext der Regulation cholinerger Prozesse in neuronalen und nicht-neuronalen Geweben. Durch die geometrische Komplexität der Interaktion von kleinen RNAs (smRNAs) mit codierenden Transkripten in der Zelle, die in der Art eines »viele-zu-viele« Netzwerkes organisiert ist, kann man eine umfassende Analyse der Dynamiken der smRNA-Reg"-u"-la"-tion nur mit bioinformatischer Hilfe unternehmen. Diese Dissertation etabliert eine bio"-in"-for"-ma"-ti"-sche Infrastruktur für solche Analysen (unter dem Namen \emph{miRNeo}), und diskutiert dann deren Anwendung an zwei klinischen Beispielen: 1) die neuronale Rolle von cholinerger smRNA-Reg"-u"-la"-tion in den psychiatrischen Erkrankungen Schizophrenie und Bipolare Störung,\cite{Lobentanzer2019a} und 2) die Relevanz cholinerg-assoziierter smRNAs in immunologischen Reflexen im Blut von Schlaganfall-Patienten.\cite{Winek2020} Im Hinblick auf die steigende Relevanz von therapeutischen Mechanismen im smRNA-Bereich gibt die Dissertation einen Ausblick auf die bioinformatisch gestützte Arzneistoff-Entwicklung, besonders in Beachtung der smRNA-typischen Funktionsweise und multiplen Interaktionen mit codierenden Genen und Transkriptionsfaktoren (TFs).

\subsection{Einführung}
Cholinerge Systeme sind unerlässlich für die physiologische Funktion des Säugetier-Organismus (Kapitel \ref{sec:intro:cholinergic}). Sie zeichnen sich aus durch die Verwendung des Signalmoleküls Acetylcholin (ACh), des ersten Neurotransmitters, für den eine derartige Funktion nachgewiesen wurde.\cite{Dale1914,Loewi1921,Dale1929} Cholinerge Systeme sind im zentralen Nervensystem hochrelevant, und hier vor allem in einer modulatorischen Rolle. Cholinerge Projektionen aus den acht primären cholinergen Kernen Ch1-Ch8 üben Kontrolle über weite Bereiche des Mittelhirns und der Hirnrinde aus.\cite{Mesulam1984} Zusätzlich existieren cholinerge Interneurone mit kurzen Projektionen in mehreren Hirngebieten, zuvorderst im Striatum, und, erst seit kurzem bekannt, in der Hirnrinde.

Cholinerge Systeme sind, parallel zu ihrer wichtigen Rolle in physiologischen Prozessen, auch beteiligt an verschiedenen Krankheitsprozessen (Kapitel \ref{sec:intro:diseases}). Ätiologisch ist eine kausale Rolle von ACh in mehreren zentralnervösen und peripheren Erkrankungen nachgewiesen, beispielsweise in der Alzheimer-Demenz, Schizophrenien und Bipolaren Störungen, sowie in peripheren Immunreflexen, auch im Zusammenhang mit Schlaganfall. In der Alzheimer-Erkrankung stellt die Hemmung des ACh-abbauenden Enzyms Acetylcholinesterase (AChE) das dominierende Therapieprin"-zip dar. Allerdings ist die aktuelle Therapie begrenzt auf Symptomatik und eventuelle Ver"-lang"-sa"-mung des geistigen Verfalls. In psychiatrischen Erkrankungen, die oft multigenetische Ätiologien aufweisen, ist die Beteiligung von cholinergen Systemen komplex und Gegenstand der aktiven Forschung. Gleichermaßen ist der Einfluss cholinerger Signale im Bereich der Immunität noch weitgehend ungeklärt. Es gibt deutliche Hinweise auf die Beteiligung von ACh an der Regulation von T-Zellen, Monozyten/Makrophagen, und an deren Mobilisierung aus der Milz. Die genauen Mechanismen sind allerdings noch kontrovers diskutiert, beispielsweise verfügt die Milz nicht über direkte vagale Innervation, was zu der Hypothese geführt hat, dass die Ausschüttung von ACh in diesem Fall von lokalen Immunzellen unter Stimulation durch den Sympathikus stattfindet (der die Milz umfassend innerviert).

Cholinerge Systeme, genauer, die Zellen der cholinergen Systeme, sind charakterisiert durch Ihre Fähigkeit zur Synthese und Freisetzung von ACh, und damit durch ihren transkriptionellen Phenotyp. Kleine RNAs (smRNAs), deren bekannteste Vertreter momentan microRNAs (miRNAs) und tRNA-Fragmente (tRFs) sind, üben auf die zelluläre Ebene des Phenotyps, bestimmt durch mRNA, kontrollierende Funktion aus (Kapitel \ref{sec:intro:mirna} \& \ref{sec:intro:trfs}). miRNAs sind in ihrem reifen Stadium kurze (etwa 18-22 Basen), einzelsträngige RNA-Moleküle, die einen Translationsstopp oder den Abbau von einzelnen mRNAs veranlassen können. Sie nutzen dafür einen Proteinkomplex, der unter dem Namen RISC (\emph{RNA-induced silencing complex}) bekannt ist, und dienen als »Zielmolekül«: durch Komplementarität mit einer kurzen Sequenz der Ziel-mRNA leiten sie den RISC zur mRNA. tRFs wurden erst vor kurzem als Kontrollmoleküle identifiziert, obwohl ihre Existenz schon länger bekannt ist. Ihre Rolle ist bisher weniger klar als die der miRNAs. Teils wurde für einige tRFs ein miRNA-Mechanismus nachgewiesen (RISC-vermittelter Translationsstopp), aber in anderen Fällen interagierten die tRFs mit anderen funktionalen Bestandteilen von Zellen, beispiels"-weise RNA-bindenden Proteinen.

Durch eine komplexe »viele-zu-viele«-Organisation ist das Netzwerk der smRNA-Kon"-trol"-le chol"-in"-er"-ger Prozesse eine Struktur, die ohne die Hilfe von informatisch gestützten Systemen vom Menschen nicht erfasst werden kann. In der Arbeit wurde ein Ansatz beschrieben, die Struktur und Dynamik von smRNAs in cholinergen Systemen sichtbar zu machen; dieser Ansatz soll jedoch nicht auf cholinerge Systeme begrenzt bleiben, sondern wurde mit dem Ziel entwickelt, jedes transkriptionelle System in jeder Zelle untersuchen zu können.

\subsection{miRNeo - eine Infrastruktur für smRNA-Dynamik}
Die Entwicklung und Pflege der interaktiven Datenbank \emph{miRNeo} macht einen großen Teil der wissenschaftlichen Arbeit an dieser Dissertation aus und zieht sich über die gesamte Dauer der Promotion (Kapitel \ref{sec:database}). Der Grundgedanke hinter \emph{miRNeo} ist die Darstellung eines komplexen bio"-lo"-gi"-schen Regulationsnetzwerkes in einer digitalen Form, die der biologischen Organisation entspricht. Digitale Datensammlungen, insbesondere Datenbanken, sind geprägt durch tabellare Organisation; die Daten werden in Zeilen und Spalten gesammelt, geordnet, und abgerufen. In vielen Fällen ist dieses Vorgehen ausreichend, denn entweder ermöglichen die Ressourcen eine effektive - wenn auch ineffiziente - Nutzung, oder die natürliche Struktur der Daten ist ohnehin tabellar, beispielsweise ein Telefonbuch. Die Interaktion von RNA-Molekülen im epitranskriptionellen Kontext jedoch ist geprägt von einer hohen Vereinzelung (die meisten potentiellen Interaktionen existieren nicht) und einer Organisationsstruktur die sich durch »viele-zu-viele« beschreiben lässt: eine miRNA reguliert mehrere (teils hunderte) mRNA-Transkripte, und ein mRNA-Transkript wird reguliert von mehreren miRNAs.

Diese strukturelle Organisation birgt eine Problematik für althergebrachte, tabellare digitale Infrastruktur: das Abfragen eines komplexen Zusammenhangs (beispielsweise die Zusammenarbeit von TFs, smRNAs, und mRNAs im cholinergen System) kann mehrere hundert oder tausend einzelne Schritte enthalten und benötigt viel Zeit, Platz, und Rechenleistung. \emph{miRNeo} dagegen basiert auf einem Graph-basierten Datenbankschema: die Einträge werden repräsentiert von Knoten (nodes) und Verbindungen zwischen diesen Knoten (edges). Diese Organisation entspricht der biologischen Organisation genauer als ein tabellares Format es könnte. Durch die technische Implementation im Rahmen der öffentlichen Datenbanksoftware »Neo4j« können so Rechenleistung, Platz, und Zeit gespart werden.

Die \emph{miRNeo}-Datenbank enthält in ihrer jetzigen Form Informationen über miRNA-Bin"-de"-ver"-hal"-ten auf der Basis sowohl von experimentell validierten als auch von bioinformatisch vorhergesagten miRNA-mRNA-Interaktionen. Durch die Integration verschiedener Quellen verfügt jede miRNA-mRNA-Verbindung über eine Wertung, die sich im Zahlenraum zwischen 3 und 20 bewegt. Durch die Verwendung einer Mindest-Wertung (in den durchgeführten Analysen meist 6 oder 7) kann das untersuchte Netzwerk nach Qualität der Information gefiltert werden. \emph{miRNeo} ent"-hält außerdem eine Sammlung von TF-Regulationskreisläufen (»regulatory circuits«), die für den menschlichen Organismus gewebespezifisch die Ermittlung der Aktivität von TFs gegenüber einzelnen Genen erlauben.\cite{Marbach2016} Zudem enthält die Datenbank eigene Vorhersagen über die miRNA-ähn"-liche Interaktion von tRFs (mittels TargetScan) und verbindet mehrere Gen-No"-men"-kla"-tur"-sys"-te"-me, um die Integration und Translation zwischen Datensätzen zu ermöglichen (und deren Fehleranfälligkeit zu reduzieren).

\emph{miRNeo} ist aber nicht nur ein Datenspeicher; Neo4j bietet mit der Sprache »CYPHER« die Möglichkeit eines effizienten Datenabrufs von komplexen Zusammenhängen innerhalb des Da"-ten"-bank-Netz"-wer"-kes. Beispielsweise können in einem Arbeitsschritt sämtliche Interaktionen von klei"-nen RNAs, die das cholinerge System betreffen, abgerufen werden, und sie können gleichzeitig mit den TFs verbunden werden, die in einem bestimmten Gewebe (oder mehreren bestimmten Geweben) den cholinergen Phänotyp steuern. Dieser Vorgang, der in einem traditionellen tabellaren System eine Kombination mehrerer Abrufe und nachfolgende Verarbeitung der großen Datenmengen im Arbeitsspeicher erfordern würde, kann durch die hohe Ähnlichkeit der biologischen Daten und der Organisation dieser Daten in \emph{miRNeo} deutlich effizienter und damit zeitsparender gestaltet werden. Dies is vor Allem von Vorteil, wenn die durchgeführten Analysen ein hohes Maß an Komplexität zeigen, beispielsweise die Ergründung von smRNA:TF:Gen »\emph{feedforward}«-Schleifen (Kapitel \ref{sec:stroke:ffl}).

\subsection{Cholinerge smRNA-Dynamik in Neuronen}
Cholinerge Neurone sind definiert durch ihre Expression von cholinergen Marker-Genen. Dies sind im engeren Sinne die Cholin-Acetyltransferase (ChAT), die zur Synthese von ACh benötigt wird, und der vesikuläre ACh-Transporter (vAChT, Gensymbol SLC18A3), der die neuronale ACh-Frei"-setz"-ung ermöglicht. Außerdem ist der hochaffine Cholintransporter (CHT-1, Gensymbol SLC5A7) fast ausschließlich in cholinergen Nervenzellen zu finden. Im Gegensatz dazu ist die Acetylcholinesterase (AChE) auch oft auf nicht-cholinergen Zelltypen zu finden. Neue Entwicklungen auf dem Gebiet des Einzelzell-Sequencing ermöglichen eine hochauflösende transkriptionelle Darstellung des zentralen Nervensystems. In Anwendung öffentlich zugänglicher Daten - gewonnen aus murinem und menschlichem Gewebe - fand eine Charakterisierung der Expressionslandschaft zentraler cholinerger Neurone statt (Kapitel \ref{sec:cellculture:singlecell}). Auf der Basis dieser Daten, genauer, der Co-Expression von cholinergen Genen und Rezeptoren für Neurokin-Signale (siehe unten), fand eine Bewertung der Eignung der Zellkultur-Modelle LA-N-2 und LA-N-5 statt. Diese menschlichen, neuronalen Zellen differenzieren nämlich unter dem Einfluss von Neurokinen zu Neuronen cholinergen Typs, erkennbar an der ansteigenden Expression von ChAT und vAChT (Kapitel \ref{sec:cellculture:model}). Neurokine sind eine Familie von Zytokin-Typ-Signalmolekülen, die ihre Funktion an Rezeptoren der Familie vom gp130-Typ (Gensymbol IL6ST) ausüben. Die bekanntesten Neurokine sind Interleukin (IL)-6, CNTF (\emph{ciliary neurotrophic factor}), und LIF (\emph{leukaemia inhibiting factor}).

Zur Untersuchung der smRNA-Dynamik in der cholinergen Entwicklung dieser Zellen unter Neurokin-Einwirkung wurden die Zellen unter der Behandlung mit CNTF zu mehreren Zeitpunkten im Differenzierungsprozess einer RNA-Analyse durch Sequenzierung aller kleinen RNAs unterzogen. RNA wurde aus LA-N-2 und LA-N-5 jeweils zu den Zeitpunkten 30 Minuten, 60 Minuten, 2 Tage, und 4 Tage isoliert und mit unbehandelten Kontrollen verglichen (Kapitel \ref{sec:cellculture:de}). Im Resultat fanden sich 490 teils drastisch veränderte miRNAs zu verschiedenen Zeitpunkten und in einer oder beiden der Zelllinien. Diese veränderten miRNAs wurden daraufhin unter Verwendung einer Annotation für miRNA Familien systematisiert und aufgrund ihrer Ziel-Proteine in funktionale Kategorien geordnet. Dabei fielen insbesondere die miRNA-Familien mir-10 und mir-199 auf, die cholinerge Kontrolle mit der Regulation von Neurokin-Prozessen verbinden. Molekularbiologische Validierungsexperimente bestätigten die cholinerge Rolle von hsa-miR-125b-5p in der Regulation der AChE. Die integrative Analyse dieser Prozesse im Kontext der transkriptionellen Veränderungen im Rahmen der Schi"-zo"-phre"-nie und der Bipolaren Störung führten zur Veröffentlichung des ersten Manuskripts.\cite{Lobentanzer2019a}

\subsection{Cholinerge smRNA-Dynamik in Immunzellen}
Das zweite Manuskript (geteilte Erstautorschaft) untersucht die Prozesse in zirkulierenden Immunzellen des Blutes, die nach einem Schlaganfall auftreten.\cite{Winek2020} Schlaganfall-Patienten aus einer multizentrischen Studie unter Leitung der Neurologie der Charité Berlin (PREDICT\cite{Hoffmann2017}) wurden für die Sequenzierung von langen und kurzen RNA-Spezies rekrutiert. Die Isolierung der RNA fand in diesem Fall aus dem Vollblut statt. Im Anschluss fand eine Vielzahl analytischer Methoden Anwendung, mit dem Ziel, die Dynamik zwischen langen und kurzen RNAs darzustellen und verständlich zu machen; manche sind in der vorliegenden Dissertation beschrieben, andere in der begleitenden Veröffentlichung, und manche sind bisher nicht veröffentlicht.

Die Ergebnisse legen eine Verschiebung der kleinen RNA-Spezies zu Ungunsten der miRNAs, aber zugunsten der tRFs nahe (Kapitel \ref{sec:stroke:descriptive}). Da die gängige Hypothese über smRNA-Funktionen einen antagonistischen Mechanismus unterstellt, bietet die gemeinsame Analyse von kurzen und langen RNAs eine interessante Perspektive auf das Zusammenspiel in den Regelkreisen der Immunzellen. Ontologische Analysen implizieren eine tiefgreifende Modulation von immunologischen Funktionen im Blut der Patienten, mit den wichtigsten Facetten »Entzündung«, »Cytokine« (einschließlich Neurokine), »Interferone«, »Apoptose«, »Koagulation«, und »Integrität von Blutgefäßen«. Zur Einordnung der Relevanz der Blutbestandteile für diese Analysen fand eine Re-Analyse von externen Datensätzen über die Expressionsunterschiede von miRNAs, tRFs, und lang"-en RNAs in verschiedenen Blutkompartimenten statt. Dabei identifizierten wir eine hohe Relevanz von CD14-positiven Monozyten für cholinerge Prozesse im Blut und beschrieben die Verteilung der in der Expression gestörten smRNAs im Blut der Patienten (Kapitel \ref{sec:stroke:celltypes}).

Von den durchgeführten Analysen wird in dieser Dissertation die Nutzung von »\emph{feedforward loops}« in der Analyse der Co-Expression von langen und kurzen RNAs (Kapitel \ref{sec:stroke:ffl}) näher be"-schrie"-ben. \emph{Feedforward loops} sind transkriptionelle Regelkreise, bestehend aus drei Teilnehmern, in diesem Fall smRNA, TF, und Gen. In diesen Regelkreisen wird die Expression eines Gens von sowohl TF als auch smRNA reguliert, wobei die smRNA auch gleichzeitig den TF beeinflusst. Eine Netzwerk-Analyse basierend auf den \emph{feedforward loops} der Schlaganfall-beeinflussten Gene identifizierte eine modulare Topographie des entstehenden Netzwerks in CD14-positiven Monozyten. Die einzelnen fünf Module wurden daraufhin auf ihre Bestandteile und auf die Funktionen, die diese Bestandteile im Kontext der Antwort auf den Schlaganfall erfüllen, untersucht. Die modul-spezifischen und inter-modularen Eigenschaften wurden grafisch und textlich zusammengefasst und umfassen (unter anderem) Regulation des Immunsystems durch Cytokine und Interferone, apoptische Prozesse und vaskuläre Permeabilität. Dabei reproduziert diese Analyse die Erkenntnisse, die aus der ontologischen Analyse der differenziellen Expression gewonnen wurden (siehe oben), doch die Applikation der \emph{feedforward loops} und anschließende Modularisierung erhöhte die »Auflösung« der ontologischen Resultate signifikant (Kapitel \ref{sec:stroke:resolution}).

\subsection{Fazit}
Die vorliegende Dissertation bringt Fortschritte im Bereich der integrativen Analyse von RNA-Mechanismen, besonders im Bereich der Regulationsnetzwerke. Der umfassendste Beitrag ist die ent"-wickelte Infrastruktur zur effizienten Berechnung komplexer epi-transkriptioneller Vorgänge zwischen kleinen RNAs, Transkriptionsfaktoren und proteincodierenden Genen. Außerdem demonstriert die Arbeit die Anwendung dieser Methodik auf aus menschlicher neuronaler Zellkultur und dem Blut von Schlaganfall-Patienten gewonnenen RNA-Messungen. Ferner identifiziert die Arbeit cholinerg relevante smRNA-Spezies, die in der Differenzierung cholinerger Nervenzellen sowie in der Immunabwehr wichtige Rollen spielen. Die beschriebenen neuronalen Zellmodelle erlauben eine weitere Untersuchung der mit der cholinergen Funktionalität von Neuronen assoziierten Prozesse, auch unter Betrachtung von geschlechtsspezifischen genetischen Einflüssen, beispielsweise in der weiteren Erforschung des Einflusses cholinerger Systeme auf psychiatrische Erkrankungen.

Die Arbeit identifiziert eine Neuro-Immun-Achse, die eine direkte Verbindung zwischen den Neurokinen, den cholinergen Systemen und deren smRNA-Regulatoren herstellt. In diesem sowie in größerem Kontext findet eine Charakterisierung der Analyse komplexer transkriptioneller Re"-gel"-krei"-se statt, insbesondere von \emph{feedforward loops}, deren biologischer Relevanz, und deren Nutz"-ung zur Untersuchung molekularer Zusammenhänge. Dies wird an einer Analyse der Rolle von CD14-positiven Monozyten im Blut von Schlaganfall-Patienten praktisch verbildlicht. Die Nutzung von komplexen Netz"-werk"-ana"-ly"-sen zur Systematisierung molekularbiologischer Untersuchungen zeigt einen Weg zum weiteren umfassenden Verständnis der epigenetischen Regelkreise auf, welches die Grundlage für eine schlussendliche Anwendung dieser Prinzipien in der Therapie von psychiatrischen und immunologischen Erkrankungen darstellt.

\newpage

\section{Publications}

\subsection{Peer-reviewed Manuscripts}
\noindent \hangindent=.6cm Lobentanzer S, Hanin G, Klein J, Soreq H (2019) \emph{Integrative Transcriptomics Reveals Sexually Dimorphic Control of the Cholinergic/Neurokine Interface in Schizophrenia and Bipolar Disorder.} Cell Reports 29(3):764-777.e5, doi: 10.1016/j.celrep.2019.09.017.

\noindent \hangindent=.6cm Stein C, Koch K, Hopfeld J, Lobentanzer S, Lau H, Klein J (2019) \emph{Impaired hippocampal and thalamic acetylcholine release in P301L tau-transgenic mice.} Brain Research Bulletin 152:134-142, doi: 10.1016/j.brainresbull.2019.07.014.

\subsection{Manuscripts in Submission}
\noindent \hangindent=.6cm Winek K$^\dagger$, Lobentanzer S$^\dagger$, Nadorp B, Dubnov S, Dames C, Moshitzky G, Hotter B, Meisel C, Greenberg DS, Shifman S, Klein J, Shenhar-Tsarfaty S, Meisel A, Soreq H (2020) \emph{Transfer RNA fragments replace microRNA regulators of the cholinergic post-stroke immune blockade.} Submitted for peer-review. ($\dagger$: equal contribution.)

\noindent \hangindent=.6cm Yuliani T, Lobentanzer S, Klein J (2020) \emph{Central cholinergic function in the rat streptozotocin model of Alzheimer´s disease.} Submitted for peer-review.

\subsection{Book Chapters}
\noindent \hangindent=.6cm Lobentanzer S \& Klein J (2019) \emph{Zentrales und Peripheres Nervensystem.} In Wichmann \& Fromme (Verleger), Handbuch für Umweltmedizin, Kapitel 11, ecomed Medizin, erg. lfg.

\subsection{Manuscripts with Preprint Status}
\noindent \hangindent=.6cm Lobentanzer S (2020) \emph{Scavenging of Interleukin 6 Receptor by Bioidentical Recombinant gp130 as Intervention in Covid-19 Exacerbation.} Submitted for peer-review, doi: 10.31219/osf.io/3gwmp. \url{https://dx.doi.org/10.31219/osf.io/3gwmp}

\noindent \hangindent=.6cm Lobentanzer S (2020) \emph{Why most pre-published research findings are false.} Not currently submitted for peer-review, doi: 10.31219/osf.io/29jnp \url{https://doi.org/10.31219/osf.io/29jnp}

\subsection{Invited Manuscripts}
\noindent \hangindent=.6cm Lobentanzer S (2020) \emph{miRNeo: an integrative transcriptomics infrastructure for whole-genome miRNA interaction profiling.} Invited by STAR Protocols, in preparation.

\subsection{Presentations}
\noindent \hangindent=.6cm Invited talk at Hebrew University (Jerusalem, Israel), Silberman Institute, for the course 'Recent technologies in molecular neuroscience' (by Hermona Soreq and David S. Greenberg), Topic: \emph{Understanding cholinergic regulation.} January 2018

\noindent \hangindent=.6cm Edward and Lili Safra Center of Neuroscience (ELSC) Annual Retreat (Ein Gedi, Israel), invited presentation, recipient of ELSC Annual Retreat Travel Grant. Topic: \emph{Sex-related perturbations in schizophrenia and bipolar disorder brains reflect microRNA-mediated cholinergic/neurokine interactions.} February 2019

\noindent \hangindent=.6cm Invited talk at Hebrew University (Jerusalem, Israel), Silberman Institute, for the course 'Recent technologies in molecular neuroscience' (by Hermona Soreq and David S. Greenberg). Topic: \emph{A systematic approach to tRNA fragment interactions.} February 2019

\noindent \hangindent=.6cm 16th International Symposium on Cholinergic Mechanisms (ISCM XVI, Rehovot, Israel), invited talk, recipient of Young Investigators Travel Grant. Topic: \emph{The cholinergic/neurokine interface.} December 2019

\newpage

\section{Curriculum Vitae (Academic, Including Teachers)}
\small

\subsubsection{\textbf{Education}}
\vspace{-8pt}
\hrule
\vspace{10pt}

{\bf Grimmelshausen-Gymnasium} \hfill September 2004

\emph{Abitur.} Grade 2.0. \hfill \emph{Gelnhausen} \medskip

\noindent {\bf School of Audio Engineering} \hfill 2004 - 2006

\emph{Audio Engineer (Diploma).} Grade 91\%. \hfill \emph{Frankfurt/Main} \medskip

\noindent {\bf Johann Wolfgang Goethe-University} \hfill September 2008 - Present

\emph{Pharmacy (Approbation).} \hfill \emph{Frankfurt/Main}
\vspace{-10pt}
\begin{itemize}[noitemsep, leftmargin=.5cm, label={\tiny\raisebox{.5ex}{\textbullet}}]
\item 1. Staatsexamen (state exam). Grade 1.75. Commendation for best exam. \hfill September 2010
\item 2. Staatsexamen (state exam). Grade 1.4. \hfill September 2012
\item 3. Staatsexamen (state exam). Grade 1.5. \hfill December 2013
\item Begin of PhD studies at the Institute of Pharmacology and Clinical Pharmacy. \hfill August 2015
\end{itemize}

\subsubsection{\textbf{Academic Teachers}}
\vspace{-8pt}
\hrule
\vspace{10pt}

{\bf Johann Wolfgang Goethe-University} \hfill September 2008 - Present

\emph{Pharmacy.} \hfill \emph{Frankfurt/Main}
\vspace{-10pt}
\begin{itemize}[noitemsep, leftmargin=.5cm, label={\tiny\raisebox{.5ex}{\textbullet}}]
\item Prof. Dr. Theodor Dingermann
\item Prof. Dr. Jennifer Dressman
\item Prof. Dr. Michael Karas
\item Prof. Dr. Jochen Klein
\item Prof. Dr. Jörg Kreuter
\item Prof. Dr. Walter Müller
\item Prof. Dr. Eugen Proschak
\item Prof. Dr. Manfred Schubert-Zsilavecz
\item Prof. Dr. Dieter Steinhilber
\end{itemize}

\noindent {\bf Harvard University} (via HarvardX) \hfill 2016 - 2017

\emph{Bioinformatics (PH525.5x: Introduction to Bioconductor).} \hfill \emph{Cambridge, MA}
\vspace{-10pt}
\begin{itemize}[noitemsep, leftmargin=.5cm, label={\tiny\raisebox{.5ex}{\textbullet}}]
\item Prof. Dr. Rafael Irizarry
\item Prof. Dr. Michael Love
\end{itemize}

\noindent {\bf Uppsala University} (via pharmbio.org) \hfill 2016 - 2017

\emph{Pharmaceutical Bioinformatics.} \hfill \emph{Uppsala, Sweden}
\vspace{-10pt}
\begin{itemize}[noitemsep, leftmargin=.5cm, label={\tiny\raisebox{.5ex}{\textbullet}}]
\item Prof. Dr. Jarl Wikberg
\item Dr. Maris Lapins
\end{itemize}

