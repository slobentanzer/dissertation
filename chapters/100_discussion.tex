%!TEX root = ../dissertation.tex
\begin{savequote}[75mm]
If the human brain were so simple that we could understand it, we would be so simple that we couldn’t.
\qauthor{Emerson M. Pugh}
\end{savequote}

\chapter{Discussion}

\newthought{Lorem ipsum dolor sit amet}, consectetuer adipiscing elit. Morbi commodo, ipsum sed pharetra gravida, orci magna rhoncus neque, id pulvinar odio lorem non turpis. Nullam sit amet enim. Suspendisse id velit vitae ligula volutpat condimentum. Aliquam erat volutpat. Sed quis velit. Nulla facilisi. Nulla libero. Vivamus pharetra posuere sapien. Nam consectetuer. Sed aliquam, nunc eget euismod ullamcorper, lectus nunc ullamcorper orci, fermentum bibendum enim nibh eget ipsum. Donec porttitor ligula eu dolor. Maecenas vitae nulla consequat libero cursus venenatis. Nam magna enim, accumsan eu, blandit sed, blandit a, eros.

\section{Methods} \label{sec:discussion:methods}
cell model, chat anomaly, regulation of expression of these two, induction, low vs high control genes
Quisque facilisis erat a dui. Nam malesuada ornare dolor. Cras gravida, diam sit amet rhoncus ornare, erat elit consectetuer erat, id egestas pede nibh eget odio. Proin tincidunt, velit vel porta elementum, magna diam molestie sapien, non aliquet massa pede eu diam. Aliquam iaculis. Fusce et ipsum et nulla tristique facilisis. Donec eget sem sit amet ligula viverra gravida. Etiam vehicula urna vel turpis. Suspendisse sagittis ante a urna. Morbi a est quis orci consequat rutrum. Nullam egestas feugiat felis. Integer adipiscing semper ligula. Nunc molestie, nisl sit amet cursus convallis, sapien lectus pretium metus, vitae pretium enim wisi id lectus. Donec vestibulum. Etiam vel nibh. Nulla facilisi. Mauris pharetra. Donec augue. Fusce ultrices, neque id dignissim ultrices, tellus mauris dictum elit, vel lacinia enim metus eu nunc.

\section{The Cholinergic/Neurokine Interface}
Hypothesis: cholinergic and neurokine systems intermingle significantly in the cns, affecting physiological as well as pathogenic (pathologic?) processes. Multiple angles reject null (orthogonal evidence)

\section{Small RNA Therapeutics and Pharmacology} \label{sec:discussion:therapy}
Extant approaches, methods, diseases, PCSK9, asthma, using small RNA antisense as substitute for single-target small molecules, reduce off-target effects, side effects of a different kind

Transcriptomics as basis for selection and design of antisense therapy, combinatorial, compare dirty drugs from psychiatric disorders, serendipity impossible, determinant is the sequence as opposed to functional groups that can be iteratively modified (only 4 building blocks)
