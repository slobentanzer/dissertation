%!TEX root = ../dissertation.tex
\begin{savequote}[75mm]
If the human brain were so simple that we could understand it, we would be so simple that we couldn’t.
\qauthor{Emerson M. Pugh}
\end{savequote}

\chapter{Discussion}
\section{Methods} \label{sec:discussion:methods}
database, approach, compare to new parallel developments (mirDIP)

graph as main advantage for dynamic analyses such as FFLs

TFs from marbach only binding, not induction/repression

FFLs: only feasible with smRNA as X, associations TF$\to$smRNA (HDAC7/RARA and miRNA-10a) still anecdotal 

FFLs might be useful to indicate induction/repression behaviour in combination with TF data

cell model, chat anomaly, regulation of expression of these two, induction, low vs high control genes

gene ontology as a tool, interpretation, confirmation bias

modularisation based on FFLs is arbitrary (resolution)

\section{The Cholinergic/Neurokine Interface}
Hypothesis: cholinergic and neurokine systems intermingle significantly in the cns, affecting physiological as well as pathogenic (pathologic?) processes. Multiple angles reject null (orthogonal evidence)

Bring together diseases in introduction

feedforward loops: tRFs ample targeting in modules, but minor in ffls... implications? less TF targeting, more "end user" gene targeting?

\section{Small RNA Therapeutics and Pharmacology} \label{sec:discussion:therapy}
Extant approaches, methods, diseases, PCSK9, asthma, using small RNA antisense as substitute for single-target small molecules, reduce off-target effects, side effects of a different kind

Transcriptomics as basis for selection and design of antisense therapy, combinatorial, compare dirty drugs from psychiatric disorders, serendipity impossible, determinant is the sequence as opposed to functional groups that can be iteratively modified (only 4 building blocks)

elusive small molecular drugs, e.g. for IRF5 \cite{Almuttaqi2019}

Monocytosis caused by either congenital ApoE KO or repeated LPS injections led to an imbalance between pro-inflammatory and wound healing monocyte types, in which increased pro-inflammatory monocytes hampered the transition to an anti-inflammatory healing microenvironment, in a murine model of myocardial infarction.\cite{Panizzi2010}

many transcription factors show highly context-dependent actions, eg MAF family \cite{Hamada2020}
