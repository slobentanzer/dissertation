%!TEX root = ../dissertation.tex
\begin{savequote}[75mm]
If the human brain were so simple that we could understand it, we would be so simple that we couldn’t.
\qauthor{Emerson M. Pugh}
\end{savequote}

\chapter{Discussion}

\newthought{Lorem ipsum dolor sit amet}, consectetuer adipiscing elit. Morbi commodo, ipsum sed pharetra gravida, orci magna rhoncus neque, id pulvinar odio lorem non turpis. Nullam sit amet enim. Suspendisse id velit vitae ligula volutpat condimentum. Aliquam erat volutpat. Sed quis velit. Nulla facilisi. Nulla libero. Vivamus pharetra posuere sapien. Nam consectetuer. Sed aliquam, nunc eget euismod ullamcorper, lectus nunc ullamcorper orci, fermentum bibendum enim nibh eget ipsum. Donec porttitor ligula eu dolor. Maecenas vitae nulla consequat libero cursus venenatis. Nam magna enim, accumsan eu, blandit sed, blandit a, eros.

\section{Methods} \label{sec:discussion:methods}
database, approach, compare to new parallel developments (mirDIP)
graph as main advantage for dynamic analyses such as FFLs

cell model, chat anomaly, regulation of expression of these two, induction, low vs high control genes


\section{The Cholinergic/Neurokine Interface}
Hypothesis: cholinergic and neurokine systems intermingle significantly in the cns, affecting physiological as well as pathogenic (pathologic?) processes. Multiple angles reject null (orthogonal evidence)

Bring together diseases in introduction

feedforward loops: tRFs ample targeting in modules, but minor in ffls... implications? less TF targeting, more "end user" gene targeting?

\section{Small RNA Therapeutics and Pharmacology} \label{sec:discussion:therapy}
Extant approaches, methods, diseases, PCSK9, asthma, using small RNA antisense as substitute for single-target small molecules, reduce off-target effects, side effects of a different kind

Transcriptomics as basis for selection and design of antisense therapy, combinatorial, compare dirty drugs from psychiatric disorders, serendipity impossible, determinant is the sequence as opposed to functional groups that can be iteratively modified (only 4 building blocks)
