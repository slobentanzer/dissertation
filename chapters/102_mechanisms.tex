%!TEX root = ../dissertation.tex
\section[A Mechanistic Perspective of Transcriptional Interactions]{A Mechanistic Perspective of \\Transcriptional Interactions}

This dissertation was aimed at elucidating epi-transcriptional processes surrounding expression of cholinergic genes. To this end, a framework was developed to assess interactions between players on the field of RNA-related processes in mammalian cells. This framework was then applied to state-of-the-art measurements of RNA levels, i.e., RNA-sequencing. In the following paragraphs, an assessment will be held on the outcomes of my efforts in clarifying »Small RNA Dynamics in Cholinergic Systems«.

\subsection{Analysis of Small RNA Dynamics via RNA-sequencing \\and Bioinformatics}
Small RNAs and the mechanisms with which they control the expression of coding genes have fascinated researchers since their discovery around the turn of the millennium. Much of the pioneering work has been done on miRNAs, but with tRFs, a new class of regulatory small RNA is increasingly being investigated in physiological and pathological contexts. However, during the initial phase of my work on the project, it quickly became apparent that an integrative view is crucial for the analyses to have any relevance to the actual processes I was trying to model. First and foremost, this involves information on the coding genes that are the supposed targets of small RNA intervention, but also the workings of transcription factors, which shape the phenotype of the cell, and, relatedly, tissue specificity of all of the aforementioned processes.

As such, a comprehensive integrative model of smRNA interactions did not exist at the time of the start my work on this dissertation, and still, \emph{miRNeo} seems to be the only effort integrating all of the processes listed above. There have been developments of integrative databases which model miRNA$\to$gene interaction, one of them also including tissue specificity and transcription factors. The most extensive efforts in my view are mirDIP, miRWalk 3.0, and miRNet (the cause of renaming of my own database). Thus, they will be briefly reviewed and compared to my own work in the following.

mirDIP 4.1\cite{Tokar2018} and miRWalk 3.0\cite{Sticht2018} are similar in their focus on miRNA$\to$gene interactions. To this end, both collected and integrated third-party data into their database. Both offer public access through a browser-based interface and database downloads. In addition mirDIP offers integration into development environments via Java, R, and Python APIs. The main difference between the two is their data aggregation approach. While the mirDIP team collected all resources available (75 different sources\cite{Tokar2018}), the miRWalk developers reduced their source count from 2.0 to 3.0, from 12 sources to 4.\cite{Sticht2018} Instead of combinatorial power, the authors of miRWalk 3.0 rely on a single algorithm as core principle of miRNA$to$gene targeting, TarPmiR.\cite{Ding2016} Briefly, TarPmiR utilises machine learning (random forest) to identify miRNA binding sites by characteristics learned from photoactivatable-ribonucleoside-enhanced crosslinking and immunoprecipitation (PAR-CLIP) sequencing results. A comparison to other prediction algorithms indicated superior performance in the authors' hands.\cite{Ding2016} However, it also shows how incomplete these approaches still are: the average recall of TarPmiR in the initial publication was 0.543, and the precision was merely 0.181 (or 0.191, the numbers in the manuscript conflict), indicating a high number of false positives. In light of these numbers, reliance on any one algorithm still remains statistically inferior to the combination of predictions based on different modelling techniques.\cite{Witkos2011} In light of the reliability of prediction algorithms, which ranges from very low to medium at best, stringent assessment of statistical properties of these data collections is necessary. 
However, the authors of miRWalk 3.0 have not statistically evaluated the performance of their database in the most recent publication.\cite{Sticht2018}

At the other extreme, mirDIP 4.1 includes 30 publicly available sources, selected from a review of 75 sources, to yield a total amount of 150 million targeting predictions.\cite{Tokar2018} Of note, due to performance issues, the database can supply only miRNA:gene interactions, without additional information such as mRNA binding site, and does not work with gene identifiers other than HUGO symbol (which may be ambiguous). For integration of all third-party datasets, the authors normalised confidence levels of predictions inside each dataset, to give a score between 0 and 1, and then ranked each prediction dataset based on a benchmarking procedure (from experimentally validated interactions). These ranks were then used to calculate the confidence of miRNA$\to$gene relationships via an integrative scoring, similar to the score in \emph{miRNeo}, but with the addition of a weight for each prediction dataset. The addition of a weight may be beneficial the more source datasets are used, to differentiate between different qualities of source material. What I did by excluding the two poor performance datasets (Section \ref{sec:database:mirna}) equals a simplified weighing procedure (with score of 1 for included datasets and 0 for dropped). A comparison of \emph{miRNeo} accuracy compared to the mirDIP 4.1 data using the benchmarking data will be interesting, but has not been performed yet. Once the mirDIP data is integrated in \emph{miRNeo}, the benefit of the weighing procedure can be evaluated.

\emph{miRNeo} is not designed to compete with these types of database; on the contrary, \emph{miRNeo} relies on a combination of publicly available datasets to enable accurate prediction\cite{Witkos2011} and to be able to derive test statistics from comparison of different source materials. Rather, \emph{miRNeo} is designed to be efficient in managing complex computations on RNA-based interactions so as to enable the study of complex relationships and biological mechanisms such as feedforward loops.

As such, miRNet\cite{Fan2016} is closest in functionality to \emph{miRNeo}, as it provides interaction data on transcription factors as well. Very recently, it seems to have been updated to version 2.0, which allows study of transcription factors and feedforward loops, but there has been no publication detailing the results as of yet (April 2020). Its main advantage (see below) over \emph{miRNeo} for users is that it provides an easy-to-use web-based interface for analyses; its main downside is that it practically includes only two main miRNA targeting sources, TarBase and miRecords. miRNA:TF data were collected from a dedicated source, TransmiR, which includes only manually curated interactions, and thus likely underestimates the true interactions by orders of magnitude. Additionally, the new version of miRNet seems to diverge significantly from the original description,\cite{Fan2016} and the only way of evaluating the database is by the very limited »About«-section on the webpage, which unfortunately features several inconsistencies. For instance, the authors state that »miRNA to TF interaction data were collected from TransmiR 2.0«, however, TransmiR is a TF$\to$miRNA database, which is critically and fundamentally different in its implications. Thus, while the idea behind miRNet may be similar to \emph{miRNeo}, function and performance cannot currently be assessed without additional information on what exactly miRNet does, and what it is based on. It may even be dangerous to present researchers with such an easily accessible tool, which from the input of only a few gene names generates complex analyses without requiring any understanding from the researcher performing the analyses.

In summary, \emph{miRNeo} as an integrative approach to small RNA dynamics is a valuable addition to the repertoire of the study of transcriptional interactions. It collects several resources for miRNA$\to$gene targeting, which have been statistically evaluated as to their performance; it integrates this targeting data with tissue-specific TF$\to$gene targeting information from 394 human tissues, based on the FANTOM5 dataset; and it provides, through its graph-based infrastructure, high computational performance for the assessment of complex relationships in RNA interaction. From these points, it is the most complete integrative transcriptional interaction database to date.

\subsection{The Cholinergic/Neurokine Interface}
Hypothesis: cholinergic and neurokine systems intermingle significantly in the cns, affecting physiological as well as pathogenic (pathologic?) processes. Multiple angles reject null (orthogonal evidence), application to disease

identified two related families, mir-10 and mir-199 as an interface
 
 co-expression in single cell
 
%Dantzer complementarity between long and short range communication
%paracrine, endocrine, LIF induces catecholaminergic-to-cholinergic switch

cell model, chat anomaly, regulation of expression of these two, induction, low vs high control genes

broadly acting vs specific mir families



Bring together diseases in introduction

\subsection{Tissue Specificity of Small RNA Species} \label{sec:discussion:tissue}

\subsection{Molecular Biology of Feedforward Loops}

feedforward loops: tRFs ample targeting in modules, but minor in ffls... implications? less TF targeting, more »end user« gene targeting?

central cluster - cooperative zone? genes involved?

mRNA down, tRF up, miRNA down -> miRNA-like tRF function coherent, miRNAs in general incoherent, displacement tRF mechanism also incoherent. but, few FFLs in miRNA-like tRFs. fig \ref{fig:stroke-de-tsne}

The relevant genes identified above are involved in 3.5\% of miRNA FFLs (681 FFLs) and 11\% of tRF FFLs (21 FFLs). 

FFLs might be useful to indicate induction/repression behaviour in combination with TF data
