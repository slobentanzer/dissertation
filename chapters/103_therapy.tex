%!TEX root = ../dissertation.tex
\section{Small RNA Therapeutics and Pharmacology} \label{sec:discussion:therapy}
Extant approaches, methods, diseases, PCSK9, asthma, using small RNA antisense as substitute for single-target small molecules, reduce off-target effects, side effects of a different kind, what are off-target-effects in miRNA therapy?

Transcriptomics as basis for selection and design of antisense therapy, combinatorial, compare dirty drugs from psychiatric disorders, serendipity impossible, determinant is the sequence as opposed to functional groups that can be iteratively modified (only 4 building blocks)

elusive small molecular drugs, e.g. for IRF5 \cite{Almuttaqi2019}

immunity: sustained central inflammation and delayed resolution, cholinergic signalling in immune cells, neurokines, TH1 type cytokines, cell exchange with CNS

stroke: monocytes from acute and subacute to chronic - cross differentiation, migration, controlled by ?

Monocytosis caused by either congenital ApoE KO or repeated LPS injections led to an imbalance between pro-inflammatory and wound healing monocyte types, in which increased pro-inflammatory monocytes hampered the transition to an anti-inflammatory healing microenvironment, in a murine model of myocardial infarction.\cite{Panizzi2010}

many transcription factors show highly context-dependent actions, eg MAF family \cite{Hamada2020}

sexual differences