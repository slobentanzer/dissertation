%!TEX root = ../dissertation.tex
\section{Small RNA Therapeutics and Pharmacology} \label{sec:discussion:therapy}
Application of small oligonucleotides in therapy of human diseases is in the early stages of development. Because of their chemical nature, i.e., high molecular weight and poly-ionised backbone, oligonucleotides cannot cross biological membranes via passive diffusion and therefore have to be delivered using pharmaceutical technology, most commonly, lipid nanoparticles.\cite{Akhtar2007,Whitehead2009} Once they have reached the cytoplasm of the target cell, therapeutic antisense oligonucleotides can load into the \acf{risc} and convey their translational suppression similar to endogenous molecules such as miRNAs and tRFs. Notably, and perhaps surprisingly, single-dose application of comparatively low doses of synthetic oligonucleotides can continuously suppress target mRNA synthesis, and consequently protein expression, over a period of months.\cite{Raal2020} The most advanced lipid nanoparticles are composed of ionisable amino lipids that self-assemble into particles of the size of approximately \SI{100}{\nano\metre} when mixed with polyanionic oligonucleotides (i.e., the drug molecules).\cite{Akhtar2007} The dual function of these amino lipids is 1) to interact with drug molecules via ion-ion interaction, forming the delivery particles and 2) to allow the drug molecules to escape from endosomes after endocytosis by the target cell. Through developments in the last two decades, these particles have reached a therapeutic index suitable for human therapy.\cite{Jayaraman2012,Raal2020}

The first FDA-approved antisense drug was afovirsen, approved in 1991 for use in human papillomavirus treatment, targeting the \emph{E2} gene implicated in virus replication. However, afovirsen and the Bcl2-antisense oblimersen failed their clinical trials. It took several more years for the first synthetic oligonucleotide to be approved for human treatment: fomivirsen, also a blocker of viral RNA, for the local treatment of HIV-associated cytomegalovirus retinitis, was approved in 1998.\cite{Piascik1999} It has since been retracted, but several others are currently approved for treatment: mipomersen for the treatment of familial hypercholesterinaemia, defibrotide for treatment of veno-occlusive liver disease, eteplirsen and golodirsen for treatment of Duchenne muscular atrophy, pegaptanib for age-related macular atrophy, nusinersen for treatment of spinal muscular atrophy, inotersen for the treatment of heritable transthyretin-mediated amyloidosis, and volanesorsen for treatment of hypertriglyceridaemia, familial chylomicronaemia syndrome and familial partial lipodystrophy.\cite{Sharad2019,Wang2020} 

Notably, extant antisense approaches are characterised by their high specificity for an affected organ (liver, eye) and a bias for rare diseases, many of them previously untreatable. The organ specificity can be explained by the delivery aspect: due to their chemical nature, they are most effective if applied to the organ directly (eye) or if they are hybridised to a targeting molecule. The discovery of hybridisation of an oligonucleotide to N-acetylgalactosamine, which specifically binds to asialo-glycoprotein receptors expressed by hepatocytes, has led to a surge in candidates for the treatment of liver-associated diseases.\cite{Wang2020} The remainder of extant approaches are explained largely by the orphan status of the targeted diseases, facilitating approval. Most pipeline drugs are being developed in the fields of oncology and neurology; as of late, RNA antisense therapeutics seem to have reached the point of profitability.\cite{Wang2020} 

Another notable common characteristic of antisense therapeutics on the market or in the pipeline is the monogenetic or pseudo-monogenetic nature of the targeted diseases. For instance, in neurology, phase III pipeline candidates include IONIS-HTT$_{Rx}$ for treatment of Huntington's disease, and tofersen for treatment of SOD1-driven amyotrophic lateral sclerosis. Both diseases are characterised by a clear understanding of how changes in single transcripts cause pathology, and thus are easily accessible to the design of an antisense sequence. Inclisiran, also known as ALN-PCSsc, has recently successfully completed phase II trial for the treatment of familial hypercholesterinaemia.\cite{Raal2020} Its target is the \ac{pcsk} mRNA, reduction of which leads to a reduction in \ac{ldl} particles. Similarly to other liver-targeted oligonucleotides, inclisiran is hybridised to  triantennary N-acetylgalactosamine carbohydrates, conveying liver-specific receptor-mediated endocytosis. The encapsulating lipid nanoparticles utilise the aminolipid DLin-MC3-DMA.\cite{Jayaraman2012} Inclisiran has been shown to significantly alleviate familial hyper"-chol"-es"-ter"-in"-aemia (\ac{ldl}-levels reduced approximately 40\%) via only bi-annual subcutaneous application.\cite{Frank-Kamenetsky2008,Fitzgerald2014,Fitzgerald2017,Raal2020}

The use of single-target oligonucleotides parallels the dogma in modern medicinal chemistry of creating drugs with as little off-target effects as possible, to be able to tightly control the biological effects of the drug while simultaneously preventing adverse effects. Due to their extreme target specificity caused by mRNA complementarity, antisense oligonucleotides mainly cause adverse effects via their application (e.g., local effects at the subcutaneous injection site) or the compounds needed to facilitate their delivery (e.g., the lipid nanoparticles). Considering the very infrequent application as compared to other subcutaneous therapeutics, the general adverse effect risk of antisense therapy can be considered low.\cite{Raal2020}

The main problem in antisense therapy in its current state is the limitation to easily accessible compartments and the single-target nature of the drugs. All diseases featured in this dissertation are known for their polygenetic or poly-factorial nature, and thus, monogenetic therapeutic approaches are bound to fail. Additionally, psychiatric diseases present the major delivery hurdle of the blood-brain-barrier, and possibly advanced delivery aspects such as single affected CNS cell types. Traditional small molecule therapeutics in psychiatric disease are known for their extreme range of target molecules. Most are derived from chlorpromazine, an artefact of antihistamine discovery synthesis, and are notorious for their targeting of multiple classes of neurotransmitter receptors. Impressively, second generation antipsychotics, which generally are seen as an improvement over the first generation of chlorpromazine-type antipsychotics, often present a more extensive and complex receptor profile, contrary to the specificity dogma of medicinal chemistry.

Translation of the knowledge of these dynamics in psychiatric diseases to antisense oligonucleotide therapy is the biggest hurdle for the design of adequate drug molecules. As opposed to small molecule drugs (high throughput screening of synthetic derivative molecules), serendipity is all but impossible in the case of antisense drugs, which are comprised of combinations of only four principal building blocks, as opposed to thousands of different chemical moieties, but a greater combinatorial multitude of possible molecules ($4^n$). The iterative screening of all possible combinations of the four bases in common high-throughput assays would quickly exceed experimental capacities and would lead to uneconomic development costs, particularly if seeking a multi-target oligonucleotide. Thus, bioinformatic predictions of suitable candidate molecules to be tested are necessary for efficient and economic screening of drug candidates for any given application, as well as for the prior identification of suitable combinations of target molecules in any given disease.

The dissertation here presented provides an infrastructure for these analysis steps (Chapter two) as well as examples for high-prevalence diseases (Chapters three and four). Integrative transcriptomics analyses can serve as tools for the identification of pertinent pathways in pathogenesis as well as for the development of oligonucleotides with multi-target behaviour that enables synergistic effects in therapy and an imitation of multiple-target small molecule drug behaviour. Priorities in these analyses can be set to reflect the researchers' focus, for instance, an analytical prior could bias the search towards pharmacologically interesting targets that are so far inaccessible to small molecule drugs, such as IRF5.\cite{Almuttaqi2019}

In immunology, hybridisation of oligonucleotide drugs to immune cell-specific receptor ligands could be used to convey a cell-type specificity akin to current liver-specific approaches. Many transcription factors show highly context-dependent activities,\cite{Hamada2020} and thus, a combination of TF specificity conveyed by the oligonucleotide and cell type specificity conveyed by a hybridisation partner may allow context-dependent intervention. For instance, ligands specific for CD4$^+$ cells could be used to target oligonucleotides to T helper cells, targeting relevant TFs such as NF-$\upkappa$B or STATs to modify the inflammatory reflex; oligonucleotides hybridised to CD14$^+$-targeting ligands could be utilised to interfere with monocyte responses, e.g. after stroke. Due to the trafficking of immune cells between brain and periphery, therapy of diseases with neuroinflammatory components could be accessible through the much easier peripheral (e.g., subcutaneous) application. A knockdown of few select monocyte-specific transcripts could be used to drive differentiation towards the pro-resolving M2-type macrophage population,\cite{Panizzi2010} possibly breaking the vicious cycle of protracted inflammation. Co-targeting of cholinergic and neurokine transcripts may interfere more causally in pathogenesis of diseases with cholinergic participation than current single-target small molecule approaches.

%sexual differences