%!TEX root = ../dissertation.tex
% the abstract

%Natural science has only just begun illuminating the immensely complex processes surrounding transcription. Epitranscriptional regulation includes, but is not limited to, transcription factors (TFs) and the recently discovered subtle fine-tuning of expression by small RNA (smRNA), including microRNAs (miRNAs) and transfer RNA fragments (tRFs). As opposed to the fairly well-characterised function of TFs in shaping the phenotype of the cell, the effects and mechanism of action of smRNA species is less understood. In particular, the multi-leveled combinatorial interaction (many-to-many) of smRNAs presents new challenges to molecular biology. This dissertation contributes to the study of smRNA dynamics in mammalian cells in several ways: 1) The exhaustive analysis of the many-to-many network of smRNA regulation cannot take place without bioinformatic support. Here, I describe the development of an integrative database (\emph{miRNeo}) capable of fast and efficient computation of complex multi-leveled transcriptional interactions. This infrastructure is then applied to two main use cases. 2) To elucidate smRNA dynamics of cholinergic systems and their relevance to psychiatric disease, an integrative transcriptomics analysis is performed on patient brain sample data, single-cell sequencing data, and two closely related \emph{in vitro} human cholinergic cellular models reflecting male and female phenotypes. 3) The dynamics between small and large RNA transcripts in the blood of stroke victims are analysed via a combination of sequencing, analysis of sorted blood cell populations, and bioinformatic methods based on the \emph{miRNeo} infrastructure. Particularly, the importance and practicality of smRNA:TF:gene feedforward loops is assessed. 

%In both analytic scenarios, I identify the most pertinent regulators of disease-relevant processes and biological pathways implicated in either pathogenesis or responses to the disease. While the examples described in chapters three and four of this dissertation are disease-specific applications of \emph{miRNeo}, the database and methods described have been developed to be applicable to the whole genome and all known smRNAs.

Quisque facilisis erat a dui. Nam malesuada ornare dolor. Cras gravida, diam sit amet rhoncus ornare, erat elit consectetuer erat, id egestas pede nibh eget odio. Proin tincidunt, velit vel porta elementum, magna diam molestie sapien, non aliquet massa pede eu diam. Aliquam iaculis. Fusce et ipsum et nulla tristique facilisis. Donec eget sem sit amet ligula viverra gravida. Etiam vehicula urna vel turpis. Suspendisse sagittis ante a urna. Morbi a est quis orci consequat rutrum. Nullam egestas feugiat felis. Integer adipiscing semper ligula. Nunc molestie, nisl sit amet cursus convallis, sapien lectus pretium metus, vitae pretium enim wisi id lectus. Donec vestibulum. Etiam vel nibh. Nulla facilisi. Mauris pharetra. Donec augue. Fusce ultrices, neque id dignissim ultrices, tellus mauris dictum elit, vel lacinia enim metus eu nunc.


