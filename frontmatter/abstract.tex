%!TEX root = ../dissertation.tex
% the abstract

Natural science has only just begun illuminating the immensely complex processes surrounding transcription. Epitranscriptional regulation includes, but is not limited to, transcription factors (TFs) and the recently discovered subtle fine-tuning of expression by small RNA (smRNA), including microRNAs (miRNAs) and transfer RNA fragments (tRFs). As opposed to the fairly well-characterised function of TFs in shaping the phenotype of the cell, the effects and mechanism of action of smRNA species is less well understood. In particular, the multi-leveled combinatorial interaction (many-to-many) of smRNAs presents new challenges to molecular biology. This dissertation contributes to the study of smRNA dynamics in mammalian cells in several ways, presented in three main chapters: 

1) The exhaustive analysis of the many-to-many network of smRNA regulation cannot take place without bioinformatic support. Here, I describe the development of an integrative database capable of fast and efficient computation of complex multi-leveled transcriptional interactions (\emph{miRNeo}). This infrastructure is then applied to two use cases. 2) To elucidate smRNA dynamics of cholinergic systems and their relevance to psychiatric disease, an integrative transcriptomics analysis is performed on patient brain sample data, single-cell sequencing data, and two closely related \emph{in vitro} human cholinergic cellular models reflecting male and female phenotypes. 3) The dynamics between small and large RNA transcripts in the blood of stroke victims are analysed via a combination of sequencing, analysis of sorted blood cell populations, and bioinformatic methods based on the \emph{miRNeo} infrastructure. Particularly, the importance and practicality of smRNA:TF:gene feedforward loops is assessed. 

In both analytic scenarios, I identify the most pertinent regulators of disease-relevant processes and biological pathways implicated in either pathogenesis or responses to the disease. While the examples described in chapters three and four of this dissertation are disease-specific applications of \emph{miRNeo}, the database and methods described have been developed to be applicable to the whole genome and all known smRNAs.

