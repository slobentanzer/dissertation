%!TEX root = ../dissertation.tex
% the acknowledgments section

%\vspace*{-50pt}
As a scientist, I am not only standing on the shoulders of figurative giants, but also on the shoulders of very real persons, without whom this dissertation would not have been possible. As such, I consider it an honour and a privilege to be in the position to give thanks to everybody who helped me along the way. I want to thank my parents, for always allowing me to follow my intuitions, and for giving me confidence and morals; and my brother, for always testing and enforcing these principles. I want to thank Betti, for listening, and for her valuable feedback and support; and Alex and Misel, for distractions and for being the friends that they are. Tobi and Christian, for giving me balance with music, physical activity, and consumption of alcoholic beverages; Frauke and Eunhye, for essentially the same, just more music and less of the other things; and Maurice, for helping me take my first steps in programming.

I also have the pleasure to thank my two advisors, Prof. Jochen Klein, for teaching me to think critically and for giving me the freedom to conduct my research in an unusual way, and Prof. Hermona Soreq, for being an exceptionally responsive and insightful partner in discussion, as well as a warm and welcoming human being. I want to thank my colleagues of the Frankfurt Institute of Pharmacology and of the Silberman Institute of Molecular Biology of the Hebrew University, for creating a pleasant and productive research atmosphere. In particular, my thanks go to Kasia for her tireless cooperation, and to Nadav, Coco, and Jonas, for lively discussions of life science and all things interesting. My thanks go to Dr. Estelle Bennett for her supervision in all matters conducted at the Hebrew University facility, and to the group of Dr. Jörg Halstenberg.

I also want to thank the original donors of cells that have led to the establishment of SH-SY5Y, LA-N-2, and LA-N-5 cell lines, and the authors of all open-source software I have used in my analyses, as well as the open, helpful, and progressive bioinformatics community. This thesis was typeset using \LaTeX, originally developed by Leslie Lamport and based on Donald Knuth's \TeX. The body text is set in 12 point Egenolff-Berner Garamond, a revival of Claude Garamont's humanist typeface. A template that can be used to format a PhD dissertation with this look \& feel has been released under the permissive \textsc{agpl} license, at \href{https://github.com/suchow/Dissertate}{github.com/suchow/Dissertate}.