%!TEX root = ../dissertation.tex

%\vspace*{100pt}
This dissertation comprises three main chapters, which are written in a combined \emph{methods-results-discussion} style. In the first main part, chapter two, I address the creation, maintenance, and usage of the database designed for assessing transcriptional interactions in the experimental parts. Since the creation process in itself is methodical, distinction between method and result can often not be implemented in a clear, »journal-style« manner. In chapters three and four however, that are concerned with experimental application of transcriptional interactions in cholinergic differentiation and disease, the manuscript will be consistently structured to visually distinguish the methods from results and discussion. Method-related paragraphs will be set in sans-serif font style, while non-method parts will be set in serif font. Because of the dimensions of this dissertation and the diverging topics, the non-method parts of each chapter will be in the style of combined results and discussion, to keep the immediate discussion close to the related results. Finally, there will be dedicated chapters for more broad and generalised discussion and conclusions.

At the date of submission, the majority of the contents of chapter three, the main experimental work of this dissertation, as well as most of the ideas developed in chapter two, have been published in a peer-reviewed journal.\cite{Lobentanzer2019a} Chapter four serves to illustrate my contributions to another manuscript that is currently in submission, awaiting response.\cite{Winek2020} That manuscript diverges from the contents described in chapter four mainly by the additional experiments concerned with validation of detected tRNA fragments, and it does not study feedforward loops. The development and usage of the database described in chapter two is invited for closer explanation by \emph{STAR Protocols} and pending submission.\cite{Lobentanzer2020}

This dissertation features content boxes for general information pertaining to a specific aspect (e.g., cholinergic genes). For very large or very small numbers, the scientific exponential notation (»E notation«) is used; e.g., \me{4.7}{5} reads as 4.7 x 10$^{-5}$, or 0.000047.

\newpage
\chapter*{Data Availability}
The biological data from sequencing of \emph{in vitro} experimental samples as well as stroke patient blood samples are deposited on NCBI GEO, and the individual accession numbers are available in the respective publications.\cite{Lobentanzer2019a,Winek2020} Code repositories are published on GitHub, and the web links are likewise available in the respective manuscripts. Additionally, for both main manuscripts, a webpage was designed to facilitate the access to networks and subnetworks in an interactive manner; the pages are publicly available via GitHub, and the links are likewise available from the respective manuscripts. The git repository can be reached at \url{https://github.com/slobentanzer/}. Required documents, including declarations and a summary in German language, can be found in Appendix \ref{appendix:req-doc}.