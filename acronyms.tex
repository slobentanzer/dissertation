%!TEX root = dissertation.tex
\acsetup{extra-style = paren}

\DeclareAcronym{a7}{
short = CHRNA7,
long = nicotinic acetylcholine receptor subunit $\alpha7$,
class = gene,
sort = chrn
}
\DeclareAcronym{abc}{
short = ABC,
long = ATP binding cassette,
class = abb
}
\DeclareAcronym{ach}{
short = ACh,
long = acetylcholine,
class = abb
}
\DeclareAcronym{ache}{
short = ACHE,
long = acetylcholinesterase,
class = gene
}
\DeclareAcronym{achep}{
short = AChE,
long = acetylcholinesterase,
extra = protein,
class = abb
}
\DeclareAcronym{acly}{
short = ACLY,
long = ATP citrate lyase,
class = gene
}
\DeclareAcronym{ad}{
short = AD,
long = Alzheimer's Disease,
class = abb
}
\DeclareAcronym{ago}{
short = Ago,
long = argonaute,
extra = protein,
class = abb
}
\DeclareAcronym{aif}{
short = AIF1,
long = allograft inflammatory factor 1,
extra = microglia marker protein,
class = gene
}
\DeclareAcronym{akt}{
short = AKT,
long = Serine/Threonine Kinase 1,
extra = also known as Protein Kinase B,
class = gene
}
\DeclareAcronym{ald}{
short = ALD,
long = adrenoleukodystrophy,
class = abb
}
\DeclareAcronym{ang}{
short = ANG,
long = angiogenin,
class = gene
}
\DeclareAcronym{api}{
short = API,
long = application programming interface,
class = abb
}
\DeclareAcronym{isg}{
short = ISG,
long = IFN-stimulated gene,
class = gene
}
\DeclareAcronym{atf3}{
short = ATF3,
long = activating transcription factor 3,
class = gene
}
\DeclareAcronym{bad}{
short = BAD,
long = BCL-2-associated agonist of cell death,
class = gene
}
\DeclareAcronym{bche}{
short = BCHE,
long = butyryl cholinesterase,
class = gene
}
\DeclareAcronym{bd}{
short = BD,
long = Bipolar Disorder,
class = abb
}
\DeclareAcronym{bdnf}{
short = BDNF,
long = brain-derived neurotrophic factor,
class = gene
}
\DeclareAcronym{bcl2}{
short = BCL-2,
long = B cell lymphoma 2,
class = gene
}
\DeclareAcronym{bmal}{
short = BMAL1,
long = brain and muscle ARNT-like protein 1,
extra = also known as ARNTL,
class = gene
}
\DeclareAcronym{ca}{
short = CA,
long = cholinergic-associated,
class = abb
}
\DeclareAcronym{cage}{
short = CAGE,
long = 5' cap analysis of gene expression,
class = abb
}
\DeclareAcronym{camp}{
short = cAMP,
long = cyclic adenosine monophosphate,
class = abb
}
\DeclareAcronym{cars}{
short = CARS,
long = compensatory anti-inflammatory response syndrome,
class = abb
}
\DeclareAcronym{ccl}{
short = CCL,
long = chemokine (C-C motif) ligand,
class = abb
}
\DeclareAcronym{cd}{
short = CD,
long = cluster of differentiation,
long-plural-form = clusters of differentiation, 
class = abb
}
\DeclareAcronym{chat}{
short = CHAT,
long = choline acetyltransferase,
class = gene
}
\DeclareAcronym{chatp}{
short = ChAT,
long = choline acetyltransferase,
extra = protein,
class = abb
}
\DeclareAcronym{chip}{
short = ChIP,
long = chromatin immunoprecipitation,
class = abb
}
\DeclareAcronym{cids}{
short = CIDS,
long = CNS injury-induced immunodepression syndrome,
class = abb
}
\DeclareAcronym{clock}{
short = CLOCK,
long = circadian locomotor output cycles kaput,
class = gene
}
\DeclareAcronym{cns}{
short = CNS,
long = central nervous system,
class = abb
}
\DeclareAcronym{cntf}{
short = CNTF,
long = ciliary neurotrophic factor,
class = gene
}
\DeclareAcronym{cntfr}{
short = CNTFR,
long = ciliary neurotrophic factor receptor,
extra = soluble,
class = gene
}
\DeclareAcronym{colq}{
short = COLQ,
long = acetylcholinesterase collagen tail peptide,
extra = ColQ,
class = gene
}
\DeclareAcronym{cry}{
short = CRY,
long = cryptochrome,
class = gene
}
\DeclareAcronym{cxcl}{
short = CCL,
long = chemokine (C-X-C motif) ligand,
class = abb
}
\DeclareAcronym{dag}{
short = DAG,
long = directed acyclic graph,
class = abb
}
\DeclareAcronym{damp}{
short = DAMP,
long = damage-associated molecular pattern,
class = abb
}
\DeclareAcronym{de}{
short = DE,
long = differentially expressed,
class = abb
}
\DeclareAcronym{dmem}{
short = DMEM,
long = Dulbecco's modified eagle medium,
class = abb
}
\DeclareAcronym{erk}{
short = ERK,
long = extracellular-signal-regulated kinase,
class = gene
}
\DeclareAcronym{fcs}{
short = FCS,
long = fetal calf serum,
class = abb
}
\DeclareAcronym{fdr}{
short = FDR,
long = false discovery ratio,
class = abb
}
\DeclareAcronym{ffl}{
short = FFL,
long = feedforward loop,
class = abb
}
\DeclareAcronym{gc}{
short = GC,
long = glucocortocoid,
class = abb
}
\DeclareAcronym{geo}{
short = GEO,
long = Gene Expression Omnibus,
extra = NCBI,
class = abb
}
\DeclareAcronym{go}{
short = GO,
long = Gene Ontology,
class = abb
}
\DeclareAcronym{gfap}{
short = GFAP,
long = glial fibrillary acidic protein,
extra = central astrocyte marker,
class = gene
}
\DeclareAcronym{gp}{
short = gp130,
long = see IL6ST,
extra = gene,
class = abb
}
\DeclareAcronym{hacu}{
short = SLC5A7,
long = high affinity choline uptake transporter,
extra = also known as HACU,
class = gene
}
\DeclareAcronym{hladr}{
short = HLA-DR,
long = monocyte human leukocyte antigen isotype DR,
class = abb
}
\DeclareAcronym{ifn}{
short = IFN, 
long = interferon,
class = gene
}
\DeclareAcronym{il}{
short = IL, %extdash package prevents break
long = interleukin,
class = gene
}
\DeclareAcronym{ilr}{
short = IL6R,
long = interleukin 6 receptor,
extra = soluble,
class = gene
}
\DeclareAcronym{ilst}{
short = IL6ST,
long = interleukin 6 signal transducer,
extra = membrane bound; also known as gp130,
class = gene
}
\DeclareAcronym{ipsc}{
short = iPSC,
long = induced pluripotent stem cell,
class = abb
}
\DeclareAcronym{irf}{
short = IRF,
long = interferon regulatory factor,
class = abb
}
\DeclareAcronym{jak}{
short = JAK,
long = janus kinase,
class = gene
}
\DeclareAcronym{jnk}{
short = JNK,
long = JUN N-terminal kinase,
class = gene
}
\DeclareAcronym{ko}{
short = KO,
long = knockout,
class = abb
}
\DeclareAcronym{la2}{
short = LA\=/N\=/2, %extdash package prevents break
long = human neuroblastoma cell line,
extra = female,
class = abb
}
\DeclareAcronym{la5}{
short = LA\=/N\=/5, %extdash package prevents break
long = human neuroblastoma cell line,
extra = male,
class = abb
}
\DeclareAcronym{lbp}{
short = LBP, 
long = lipopolysaccharide binding protein,
class = abb
}
\DeclareAcronym{ldl}{
short = LDL, 
long = low density lipoprotein,
class = abb
}
\DeclareAcronym{ldt}{
short = LDT,
long = laterodorsal tegmentum,
extra = Ch6,
class = abb
}
\DeclareAcronym{lfc}{
short = LFC,
long = log$_2$ fold change,
class = abb
}
\DeclareAcronym{lif}{
short = LIF,
long = leukaemia inhibitory factor,
class = gene
}
\DeclareAcronym{lifr}{
short = LIFR,
long = leukaemia inhibiting factor receptor,
extra = soluble,
class = gene
}
\DeclareAcronym{loo}{
short = LOO,
long = Leave-One-Out,
extra = approach,
class = abb
}
\DeclareAcronym{lps}{
short = LPS,
long = lipopolysaccharide,
class = abb
}
\DeclareAcronym{mapk}{
short = MAPK,
long = mitogen-activated protein kinase,
class = gene
}
\DeclareAcronym{mbl}{
short = MBL,
long = mannan-binding lectin,
class = abb
}
\DeclareAcronym{mcao}{
short = MCAO,
long = middle cerebral artery occlusion,
class = abb
}
\DeclareAcronym{mcl1}{
short = MCL1,
long = myeloid leukaemia cell differentiation protein 1,
class = gene
}
\DeclareAcronym{mhc}{
short = MHC,
long = major histocompatibility complex,
class = gene
}
\DeclareAcronym{mir}{
short = miRNA,
long = microRNA,
class = abb
}
\DeclareAcronym{myd88}{
short = MyD88,
long = myeloid differentiation primary response 88,
class = gene
}
\DeclareAcronym{nd10}{
short = ND10,
long = nuclear domain 10,
class = abb
}
\DeclareAcronym{mrs}{
short = mRS,
long = modified Rankin Scale,
extra = clinical score of stroke severity,
class = abb
}
\DeclareAcronym{nfkb}{
short = NF-$\upkappa$B,
long = nuclear factor 'kappa-light-chain-enhancer' of activated B-cells,
class = gene
}
\DeclareAcronym{ngfr}{
short = NGFR,
long = nerve growth factor receptor,
extra = also known as p75,
class = gene
}
\DeclareAcronym{npas}{
short = NPAS2,
long = neuronal PAS domain protein 2,
class = gene
}
\DeclareAcronym{nr1d1}{
short = NR1D1,
long = nuclear receptor subfamily 1; group D; member 1,
extra = also known as Rev-Erb$\upalpha$,
class = gene
}
\DeclareAcronym{nr1f1}{
short = NR1F1,
long = nuclear receptor subfamily 1; group F; member 1,
extra = also known as ROR$\upalpha$,
class = gene
}
\DeclareAcronym{nt}{
short = nt,
short-plural-form = nt,
long = nucleotide,
class = abb
}
\DeclareAcronym{ntrk1}{
short = NTRK1,
long = neurotrophic receptor tyrosine kinase 1,
class = gene
}
\DeclareAcronym{ntrk2}{
short = NTRK2,
long = neurotrophic receptor tyrosine kinase 2,
class = gene
}
\DeclareAcronym{ncbi}{
short = NCBI,
long = National Center for Biotechnology Information,
class = abb
}
\DeclareAcronym{ngf}{
short = NGF,
long = nerve growth factor,
class = gene
}
\DeclareAcronym{oli}{
short = OLIG1,
long = oligodendrocyte transcription factor 1,
class = gene
}
\DeclareAcronym{or}{
short = OR,
long = odds ratio,
class = abb
}
\DeclareAcronym{pbg}{
short = PBG,
long = parabigeminal nucleus,
extra = Ch8,
class = abb
}
\DeclareAcronym{pbs}{
short = PBS,
long = phosphate buffered saline,
class = abb
}
\DeclareAcronym{pca}{
short = PCA, 
long = principal component analysis,
class = abb
}
\DeclareAcronym{pcr}{
short = RT\-/qPCR, %extdash package allows break
long = real-time quantitative polymerase chain reaction,
class = abb
}
\DeclareAcronym{pcsk}{
short = PCSK9,
long = proprotein convertase subtilisin-kexin type 9,
class = gene
}
\DeclareAcronym{pd}{
short = PD,
long = Parkinson's Disease,
class = abb
}
\DeclareAcronym{per}{
short = PER,
long = period,
class = gene
}
\DeclareAcronym{pi3k}{
short = PI3K,
long = phosphoinositide 3-kinase,
class = gene
}
\DeclareAcronym{prima}{
short = PRIMA1,
long = proline-rich membrane anchor 1,
class = abb
}
\DeclareAcronym{ppn}{
short = PPN,
long = pedunculo-pontine nucleus,
extra = Ch5,
class = abb
}
\DeclareAcronym{rbf}{
short = RBFOX3,
long = RNA-binding Fox-1 homolog 3,
extra = neuronal marker gene; also known as NeuN,
class = gene
}
\DeclareAcronym{rem}{
short = REM,
long = rapid eye movement,
class = abb
}
\DeclareAcronym{rin}{
short = RIN,
long = RNA integrity number,
extra = RNA quality measure,
class = abb
}
\DeclareAcronym{risc}{
short = RISC,
long = RNA-induced silencing complex,
class = abb
}
\DeclareAcronym{rora}{
short = RORA,
long = RAR-related orphan receptor $\upalpha$,
extra = see NR1F1,
class = gene
}
\DeclareAcronym{rpmi}{
short = RPMI1640,
long = Roswell Park Memorial Institute medium,
class = abb
}
\DeclareAcronym{scn}{
short = SCN,
long = suprachiasmatic nuclei,
class = abb
}
\DeclareAcronym{scz}{
short = SCZ,
long = Schizophrenia,
class = abb
}
\DeclareAcronym{seq}{
short = RNA-seq,
long = RNA sequencing,
class = abb
}
\DeclareAcronym{sg}{
short = SG,
long = significant gene,
extra = as in »differentially expressed«,
class = abb
}
\DeclareAcronym{sirs}{
short = SIRS,
long = systemic inflammatory response syndrome,
class = abb
}
\DeclareAcronym{slc}{
short = SLC18A3,
long = vesicular acetylcholine transporter,
extra = official gene symbol,
class = gene
}
\DeclareAcronym{smrna}{
short = smRNA,
long = small non-coding RNA,
class = abb
}
\DeclareAcronym{sql}{
short = SQL,
long = structured query language,
class = abb
}
\DeclareAcronym{sst}{
short = SST,
long = somatostatin,
class = gene
}
\DeclareAcronym{stat}{
short = STAT,
long = signal transducer and activator of transcription,
class = gene
}
\DeclareAcronym{tf}{
short = TF,
long = transcription factor,
class = abb
}
\DeclareAcronym{tgf}{
short = TGF,
long = transforming growth factor,
class = gene
}
\DeclareAcronym{tirna}{
short = tiRNA,
long = transfer RNA half,
long-plural-form = transfer RNA halves,
class = abb
}
\DeclareAcronym{tlr}{
short = TLR,
long = toll-like receptor,
class = gene
}
\DeclareAcronym{tnf}{
short = TNF,
long = tumour necrosis factor,
class = gene
}
\DeclareAcronym{tpm}{
short = TPM,
long = transcripts per million,
class = abb
}
\DeclareAcronym{trna}{
short = tRNA,
long = transfer RNA,
class = abb
}
\DeclareAcronym{trf}{
short = tRF,
long = transfer RNA fragment,
class = abb
}
\DeclareAcronym{tsne}{
short = t-SNE,
long = t-distributed stochastic neighbour embedding,
class = abb
}
\DeclareAcronym{tyk}{
short = TYK,
long = tyrosine kinase,
class = gene
}
\DeclareAcronym{utr}{
short = UTR,
long = untranslated region,
class = abb
}
\DeclareAcronym{vacht}{
short = vAChT,
long = vesicular acetylcholine transporter,
extra = protein; from SLC18A3 gene,
class = abb
}
\DeclareAcronym{vegf}{
short = VEGF,
long = vascular endothelial growth factor,
class = gene
}
\DeclareAcronym{vip}{
short = VIP,
long = vasoactive intestinal peptide,
class = gene
}
\DeclareAcronym{vp}{
short = VP,
long = vascular permeability,
class = abb
}
\DeclareAcronym{wt}{
short = WT,
long = wild type,
class = abb
}

